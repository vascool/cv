\documentclass[portuguese,a4paper]{europasscv}
\usepackage[portuguese]{babel}
\def\myname{Vasco Almeida}

\hypersetup{
	pdftitle={Curriculum vitae de \myname},
	pdfauthor={\myname},
	pdfsubject={Curriculum vitae de \myname},
	pdfkeywords={currículo, cv, europass}
}

\ecvname{\myname}
\ecvaddress{44 Rua da Igreja 9760-051 Biscoitos Portugal}
\ecvtelephone[(+351) 916 480 802]{}
\ecvemail{vascomalmeida@sapo.pt}
\ecvim{Wire}{vascoalmeida}
\ecvim{Skype}{ubunkastor}
\ecvdateofbirth{13 fevereiro 1994}
\ecvnationality{Portuguesa}
\ecvgender{Masculino}

\begin{document}
	\begin{europasscv}
	\ecvpersonalinfo

	\ecvsection{Experiência profissional}
	\ecvtitle{janeiro 2017 -- novembro 2017}{Estagiar L}
	\ecvitem{}{Caixa Económica da Misericórdia de Angra do Heroísmo}
	\ecvitem{}{
	\begin{ecvitemize}
	    \item desenvolvimento de queries SQL em SQL Server e
		    processos ETL em SQL Server Integration Services (SSIS)
	    \item atualização de documentação sobre esquema da base de dados
	    \item correções de bugs de uma aplicação interna escrita em Visual Basic
            \item Helpdesk aos utilizadores
	    \item exploração de ferramentas de Business Intelligence
	\end{ecvitemize}
	}


	\ecvsection{Educação e formação}

	\ecvtitle{2012--2015}{Licenciatura em Engenharia Informática e de Computadores}
	\ecvitem{}{Instituto Superior Técnico (IST), Lisboa Portugal}
	\ecvitem{}{
	    \begin{ecvitemize}
		\item Base de Dados
		\item Programação com Objetos
		\item Sistemas Distribuídos
		\item Engenharia de Software
		\item Compiladores
		\item Matemática
	    \end{ecvitemize}
	}


	\ecvsection{Competências pessoais}

	\ecvmothertongue{Português}
	\ecvlanguageheader
	\ecvlanguage{Inglês}{C1}{C1}{A2}{A2}{B2}
	\ecvlanguagefooter

	\ecvblueitem{Competências de comunicação}{
	\begin{ecvitemize}
		\item trabalho em equipa: experiência de equipa ganha a trabalhar em projetos académicos e profissionais
		\item capacidade interpessoal: ganha pela experiência a realizar levantamento da população da freguesia dos Biscoitos no
			âmbito do programa OTL
	\end{ecvitemize}
	}

	\ecvblueitem{Competências de organização}{
	\begin{ecvitemize}
		\item visão e postura organizacional e procedimentos internos, adquiridos ao estagiar numa instituição financeira
		\item boa capacidade de organização, ganha ao longo do percurso académico
		\item empenho em aprendizagem contínua e desenvolvimento de competências em várias áreas
	\end{ecvitemize}
	}

	\ecvblueitem{Competências relacionadas com o trabalho}{
	\begin{ecvitemize}
		\item proficiência em resolução de problemas informáticos, ganha ao lidar com problemas quotidianos
		\item capacidade de trabalho em equipa e relacionamento interpessoal, ganha em projetos realizados em equipa de desenvolvimento
		\item experiência básica em diagnóstico e reparação de computadores, ganha a reparar computadores pessoais
	\end{ecvitemize}
	}

	\ecvblueitem{Competências digitais}{
	\begin{ecvitemize}
		\item formação e experiência em programação e desenvolvimento de software
		\item bom domínio de ferramenta de suite de escritório (e.g. processador de texto, folha de cálculo)
		\item familiarização com tecnologia web (html, css, javascript, PHP, Django)
		\item versatilidade e adaptação a vários ambientes (utilização e instalação de diferentes sistemas operativos e software)
	\end{ecvitemize}
	}
	\ecvblueitem{Outras competências}{
		Músico amador em bandas filarmónicas.
	}

	\ecvblueitem{Carta de condução}{B}
	\ecvblueitem{Veículo próprio}{Sim}


	\ecvsection{Informação adicional}

	\ecvblueitem{Filiações}{
	\begin{ecvitemize}
		\item membro da Filarmónica Progresso Biscoitense
		\item membro da Fanfarra Operária Gago Coutinho e Sacadura Cabral
	\end{ecvitemize}
	}
	\end{europasscv}
\end{document}
