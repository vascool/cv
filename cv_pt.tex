\documentclass[portuguese,a4paper]{europasscv}
\usepackage[portuguese]{babel}
\def\name{Vasco Almeida}

\hypersetup{
	pdftitle={\name: Curriculum Vitae},
	pdfauthor={\name},
	pdfsubject={Curriculum Vitae},
	pdfkeywords={technology, computer science}
}

\ecvname{\name}
\ecvaddress{44 Rua da Igreja 9760-051 Biscoitos Portugal}
\ecvemail{vascomalmeida@sapo.pt}
\ecvdateofbirth{13 fevereiro 1994}
\ecvnationality{Portuguesa}
\ecvgender{Masculino}

\begin{document}
	\begin{europasscv}
	\ecvpersonalinfo

	\ecvsection{Educação e formação}

	\ecvtitle{2012--2015}{Licenciatura em Engenharia Informática e de Computadores}
	\ecvitem{}{Instituto Superior Técnico (IST), Lisboa Portugal}
	\ecvitem{}{
	    \begin{ecvitemize}
		\item Informática
		\item Matemática
	    \end{ecvitemize}
	}

	\ecvsection{Competências pessoais}
	\ecvmothertongue{Português}
	\ecvlanguageheader
	\ecvlanguage{Inglês}{C1}{C2}{B2}{B2}{C1}
	\ecvlanguagefooter

	\ecvblueitem{Competências de comunicação}{
	\begin{ecvitemize}
		\item experiência de equipa ganha a trabalhar em projetos académicos
		\item competência transversal ganha pela experiência a realizar levantamento da freguesia dos Biscoitos
	\end{ecvitemize}
	}

	\ecvsection{Proficiência em tecnologias}
	\ecvblueitem{Linguagens de programação}{C, C++, Java}
	\ecvblueitem{Sistemas operativos}{Linux}

	\ecvblueitem{Competências laborais}{
	\begin{ecvitemize}
		\item proficiência em resolução de problemas informáticos ganha ao lidar com problemas quotidianos
	\end{ecvitemize}
	}

	\ecvblueitem{Outras competências}{
		Toquei saxofone na banda filarmónica da Sociedade Recreativa Biscoitense.
	}

	\ecvblueitem{Carta de Condução}{B}

	\ecvsection{Contribuições}
	\ecvblueitem{Git - C, Shell}{
		Internacionalização (i18n) e localização (l10n)
		\url{https://github.com/git/git/commits?author=vascool}
	}


	\end{europasscv}
\end{document}
